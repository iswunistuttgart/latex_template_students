\chapter{Beispiele}

Nachfolgend sind einige Beispiele zur Umsetzung in \LaTeX{} aufgelistet. Für formelle Anforderungen werfen Sie bitte einen Blick in den Leitfaden für die Anfertigung studentischer Arbeiten am ISW.

\section{Aufzählungen}

Für eine Einführung in Listenumgebungen mit \LaTeX{} kann \url{https://en.wikibooks.org/wiki/LaTeX/List_Structures} verwendet werden.

Eine ungeordnete Aufzählung kann so aussehen:

\begin{itemize}
    \item Fusce tincidunt consectetur nisl a pretium. Nam sed eleifend nunc. Nulla feugiat nisl ac mauris varius, eu viverra tellus condimentum. Nullam tempus dolor a elementum con-vallis. Nam sagittis, nisi non tempor luctus, enim ex pretium nunc, lacinia suscipit arcu augue id sem.
    \item Fusce tincidunt consectetur nisl a pretium. Nam sed eleifend nunc. Nulla feugiat nisl ac mauris varius, eu viverra tellus condimentum. Nullam tempus dolor a elementum con-vallis. Nam sagittis, nisi non tempor luctus, enim ex pretium nunc, lacinia suscipit arcu augue id sem.
    \item Fusce tincidunt consectetur nisl a pretium. Nam sed eleifend nunc. Nulla feugiat nisl ac mauris varius, eu viverra tellus condimentum. Nullam tempus dolor a elementum con-vallis. Nam sagittis, nisi non tempor luctus, enim ex pretium nunc, lacinia suscipit arcu augue id sem.
\end{itemize}

Für eine geordnete Aufzählungnutzen Sie bitte die \texttt{enumerate}-Umgebung.

\begin{enumerate}
    \item Fusce tincidunt consectetur nisl a pretium. Nam sed eleifend nunc. Nulla feugiat nisl ac mauris varius, eu viverra tellus condimentum. Nullam tempus dolor a elementum con-vallis. Nam sagittis, nisi non tempor luctus, enim ex pretium nunc, lacinia suscipit arcu augue id sem.
    \item Fusce tincidunt consectetur nisl a pretium. Nam sed eleifend nunc. Nulla feugiat nisl ac mauris varius, eu viverra tellus condimentum. Nullam tempus dolor a elementum con-vallis. Nam sagittis, nisi non tempor luctus, enim ex pretium nunc, lacinia suscipit arcu augue id sem.
    \item Fusce tincidunt consectetur nisl a pretium. Nam sed eleifend nunc. Nulla feugiat nisl ac mauris varius, eu viverra tellus condimentum. Nullam tempus dolor a elementum con-vallis. Nam sagittis, nisi non tempor luctus, enim ex pretium nunc, lacinia suscipit arcu augue id sem.
\end{enumerate}

Beschreibungen lassen sich mit der \texttt{description}-Umgebung erreichen:

\begin{description}
    \item[Mücke] Fusce tincidunt consectetur nisl a pretium. Nam sed eleifend nunc. Nulla feugiat nisl ac mauris varius, eu viverra tellus condimentum. Nullam tempus dolor a elementum con-vallis. Nam sagittis, nisi non tempor luctus, enim ex pretium nunc, lacinia suscipit arcu augue id sem.
    \item[Emu] Fusce tincidunt consectetur nisl a pretium. Nam sed eleifend nunc. Nulla feugiat nisl ac mauris varius, eu viverra tellus condimentum. Nullam tempus dolor a elementum con-vallis. Nam sagittis, nisi non tempor luctus, enim ex pretium nunc, lacinia suscipit arcu augue id sem.
    \item[Gürteltier] Fusce tincidunt consectetur nisl a pretium. Nam sed eleifend nunc. Nulla feugiat nisl ac mauris varius, eu viverra tellus condimentum. Nullam tempus dolor a elementum con-vallis. Nam sagittis, nisi non tempor luctus, enim ex pretium nunc, lacinia suscipit arcu augue id sem.
\end{description}

\section{Zitieren}

Zitieren Sie mit dem \texttt{\textbackslash cite\{\}}-Befehl \cite{Cochran2005,Cubitt2013}.
Autoren und Titel können Sie ebenfalls zitieren -- mit \texttt{\textbackslash citeauthor\{\}} und \texttt{\textbackslash citetitle\{\}}: \citeauthor{Feuersaenger2014} entwickelte \texttt{pgfplots} und beschreibt das Paket in der Dokumentation \textit{\citetitle{Feuersaenger2014}}.

Nutzen Sie Tools zur Literaturverwaltung, wie JabRef oder Citavi.

\section{Formelsatz}

Eine Einführung in den Formelsatz mit \LaTeX{} finden Sie unter \url{https://de.wikibooks.org/wiki/LaTeX-Kompendium:_F%C3%BCr_Mathematiker}.
Auch die Wikipedia-Seite zum Formelsatz (\url{https://de.wikipedia.org/wiki/Formelsatz}) ist einen Besuch wert.

Formeln im Text lassen sich durch \texttt{\ \(\ \ldots \)} setzen, beispielsweise \(\sqrt{\alpha^2}\)) oder \( \bmat{a & b & c}^\T \) (\texttt{\textbackslash bmat} und \texttt{\textbackslash T} sind selbst definierte Makros in \texttt{settings.tex}).

Mehrzeilige Mathe-Umgebungen können z.B mit der \texttt{align}-Umgebung gesetzt werden. Ein \&-Zeichen kann dabei für die vertikale Ausrichtung genutzt werden.

\begin{align}
c^2 &= a^2 + b^2 \nonumber \\
\Leftrightarrow \quad c &= \pm \sqrt{a^2 + b^2} \label{eq:pythagoras_solution}
\end{align}


Nummerieren Sie nur Gleichungen, die später referenziert werden, wie \autoref{eq:pythagoras_solution}. 
\texttt{\textbackslash nonumber} unterdrückt das Erzeugen einer Nummer.

\section{Tabellen}


Tabellen werden, wie \autoref{tab:tierpreise} gezeigt, in einer \texttt{table}-Umgebung gesetzt.
Sie müssen immer im vorherigen Text referenziert und erläutert werden. 
Durch Einbinden des Pakets \href{https://ctan.kako-dev.de/macros/latex/contrib/booktabs/booktabs.pdf}{\texttt{booktabs}} lassen sich Tabellen einfach setzen. 

    \begin{table}[htb]
    \centering
    \begin{tabular}{@{}llr@{}} \toprule
    \multicolumn{2}{c}{Artikel} \\ 
    \cmidrule(r){1-2}Tier  & Beschreibung & Preis (€)\\ \midrule
        Mücke & pro Gramm    & 13.65 \\
        & pro Stück    & 0.01 \\
        Gnu   & ausgestopft  & 92.50 \\
        Emu   & ausgestopft  & 33.33 \\
        Gürteltier & gefroren & 8.99 \\ \bottomrule
    \end{tabular}
    \caption{Beispieltabelle mit dem Paket booktabs.} 
    \label{tab:tierpreise}
    \end{table}

\section{Grafiken}

Einfache Grafiken lassen sich mit \texttt{\textbackslash includegraphics} einbinden. Nutzen Sie für Grafiken immer eine \texttt{figure}-Umgebung. Referenzieren Sie Grafiken, bevor Sie im Text erscheinen, wie \autoref{fig:ustutt_logo}, und vergeben Sie aussagekräftige Bildunterschriften.

\begin{figure}[htb]
\centering
\includegraphics[width=0.5\textwidth]{images/logo-university-de.eps}
\caption{Das Logo der Universität Stuttgart.}
\label{fig:ustutt_logo}
\end{figure}

In \autoref{fig:blockschaltbild}, \autoref{fig:timeseries_states}, \autoref{fig:timeseries_input} und \autoref{fig:phasenplot} sind Grafiken TikZ und PGFplots abgebildet, die hilfreich für die eigene Arbeit sein können.


\begin{figure}[htbp]
    \centering
    \begin{tikzpicture}

    \node[block] (regler) {Regler\\ \(k(e)\)};
    \node[block, right=2cm of regler] (strecke) {Regelstrecke\\ \(\dot x = f(x,u) \)\\ \(y=h(x) \)};

    \node[sum, left = of regler] (err) {};

    \draw[->] (regler) -- node[above]{\(u\)} (strecke);
    \draw[->] (err) -- node[above]{\(e\)}(regler);

    \draw[->](strecke) -- node[above]{\(y\)} ++(2cm, 0);
    \draw[->] (strecke) -| ++(1.5cm,-1.2cm) -| node[right, pos=0.95]{\(-\)} (err);

    \draw[<-] (err) -- node[left, pos=1]{\(y_\mathrm{des}\)} ++(-2cm, 0);

\end{tikzpicture}
    \caption{Ein einfaches Blockschaltbild.}
    \label{fig:blockschaltbild}
\end{figure}

\begin{figure}[htbp]

\centering
\begin{subfigure}[b]{0.95\textwidth}
    % This file was created by matlab2tikz.
%
\definecolor{mycolor1}{rgb}{0.00000,0.56863,0.86275}%
\definecolor{mycolor2}{rgb}{1.00000,0.54902,0.14902}%
\definecolor{mycolor3}{rgb}{0.30588,0.74118,0.35294}%
%
\begin{tikzpicture}

\begin{axis}[%
width=0.951\figurewidth,
height=\figureheight,
at={(0\figurewidth,0\figureheight)},
scale only axis,
xmin=0,
xmax=5,
xlabel={Time \(t\) [s]},
y label style={at={(-0.112,0.5)}},
ymin=-10,
ymax=10,
ylabel={States},
axis background/.style={fill=white},
axis x line*=bottom,
axis y line*=left,
legend style={legend cell align=left,align=left,draw=white!15!black, at={(axis description cs:.89,1.3)},anchor=north}
]
\addplot [color=mycolor1,solid]
  table[row sep=crcr]{%
0	5\\
0.01	4.99524563586321\\
0.02	4.98101630642928\\
0.03	4.95737192324524\\
0.04	4.92438893489464\\
0.05	4.88216801700883\\
0.06	4.83084726513787\\
0.07	4.77062670520508\\
0.08	4.70181889185563\\
0.09	4.62496936786054\\
0.1	4.54120118871523\\
0.11	4.45324665779081\\
0.12	4.36455130558219\\
0.13	4.27708125431517\\
0.14	4.19134969607215\\
0.15	4.10733805689558\\
0.16	4.0250141878598\\
0.17	3.94434437737414\\
0.18	3.86529525509785\\
0.19	3.78783408212801\\
0.2	3.71192878308595\\
0.21	3.63754793864312\\
0.22	3.56466077249035\\
0.23	3.49323713780655\\
0.24	3.42324750393716\\
0.25	3.35466294336984\\
0.26	3.28745511901083\\
0.27	3.22159627175465\\
0.28	3.1570592083387\\
0.29	3.0938172894748\\
0.3	3.03184441824977\\
0.31	2.97111502878778\\
0.32	2.91160407516711\\
0.33	2.85328702058463\\
0.34	2.79613982676135\\
0.35	2.74013894358264\\
0.36	2.6852612989671\\
0.37	2.63148428895808\\
0.38	2.57878576803231\\
0.39	2.52714403962003\\
0.4	2.47653784683139\\
0.41	2.42694636338398\\
0.42	2.37834918472657\\
0.43	2.33072631935423\\
0.44	2.28405818031023\\
0.45	2.23832557687022\\
0.46	2.19350970640432\\
0.47	2.14959214641296\\
0.48	2.10655484673226\\
0.49	2.06438012190512\\
0.5	2.02305064371406\\
0.51	1.98254943387211\\
0.52	1.94285985686812\\
0.53	1.90396561296289\\
0.54	1.86585073133274\\
0.55	1.8284995633572\\
0.56	1.79189677604749\\
0.57	1.75602734561261\\
0.58	1.72087655116011\\
0.59	1.68642996852835\\
0.6	1.65267346424742\\
0.61	1.6195931896259\\
0.62	1.58717557496055\\
0.63	1.55540732386645\\
0.64	1.52427540772469\\
0.65	1.49376706024524\\
0.66	1.46386977214239\\
0.67	1.43457128592035\\
0.68	1.4058595907666\\
0.69	1.3777229175507\\
0.7	1.35014973392623\\
0.71	1.32312873953367\\
0.72	1.29664886130212\\
0.73	1.27069924884756\\
0.74	1.24526926996584\\
0.75	1.22034850621815\\
0.76	1.19592674860714\\
0.77	1.17199399334169\\
0.78	1.14854043768858\\
0.79	1.12555647590896\\
0.8	1.10303269527819\\
0.81	1.0809598721869\\
0.82	1.05932896832193\\
0.83	1.03813112692517\\
0.84	1.01735766912884\\
0.85	0.997000090365587\\
0.86	0.977050056851735\\
0.87	0.957499402142291\\
0.88	0.938340123756105\\
0.89	0.919564379869777\\
0.9	0.901164486078865\\
0.91	0.883132912224982\\
0.92	0.865462279287434\\
0.93	0.848145356338032\\
0.94	0.831175057557802\\
0.95	0.814544439314279\\
0.96	0.798246697298145\\
0.97	0.782275163717989\\
0.98	0.766623304551978\\
0.99	0.751284716855268\\
1	0.736253126122008\\
1.01	0.72152238370081\\
1.02	0.707086464262578\\
1.03	0.692939463319636\\
1.04	0.679075594795081\\
1.05	0.665489188641335\\
1.06	0.6521746885069\\
1.07	0.639126649450301\\
1.08	0.626339735700277\\
1.09	0.613808718461249\\
1.1	0.601528473763155\\
1.11	0.589493980354737\\
1.12	0.577700317639388\\
1.13	0.566142663652695\\
1.14	0.554816293080827\\
1.15	0.543716575318924\\
1.16	0.53283897256869\\
1.17	0.52217903797436\\
1.18	0.5117324137963\\
1.19	0.501494829621425\\
1.2	0.491462100609733\\
1.21	0.481630125776181\\
1.22	0.471994886307207\\
1.23	0.462552443911186\\
1.24	0.453298939202125\\
1.25	0.444230590115933\\
1.26	0.435343690358591\\
1.27	0.42663460788559\\
1.28	0.41809978341198\\
1.29	0.409735728952434\\
1.3	0.401539026390699\\
1.31	0.393506326077847\\
1.32	0.385634345458741\\
1.33	0.377919867726138\\
1.34	0.370359740501881\\
1.35	0.362950874544623\\
1.36	0.355690242483541\\
1.37	0.348574877577528\\
1.38	0.341601872499339\\
1.39	0.334768378144179\\
1.4	0.328071602462255\\
1.41	0.321508809314795\\
1.42	0.315077317353064\\
1.43	0.308774498919909\\
1.44	0.302597778973388\\
1.45	0.296544634032018\\
1.46	0.290612591141226\\
1.47	0.284799226860563\\
1.48	0.27910216627126\\
1.49	0.273519082003727\\
1.5	0.268047693284576\\
1.51	0.262685765002797\\
1.52	0.257431106794671\\
1.53	0.252281572147071\\
1.54	0.247235057518759\\
1.55	0.242289501479331\\
1.56	0.237442883865443\\
1.57	0.232693224953979\\
1.58	0.228038584651809\\
1.59	0.223477061701814\\
1.6	0.219006792904847\\
1.61	0.2146259523573\\
1.62	0.210332750703979\\
1.63	0.20612543440596\\
1.64	0.202002285023145\\
1.65	0.197961618511203\\
1.66	0.19400178453262\\
1.67	0.190121165781565\\
1.68	0.186318177322301\\
1.69	0.182591265940867\\
1.7	0.178938909509756\\
1.71	0.175359616365342\\
1.72	0.171851924697786\\
1.73	0.168414401953182\\
1.74	0.165045644247681\\
1.75	0.161744275793374\\
1.76	0.158508948335677\\
1.77	0.155338340601997\\
1.78	0.152231157761455\\
1.79	0.149186130895436\\
1.8	0.14620201647875\\
1.81	0.143277595871199\\
1.82	0.140411674819329\\
1.83	0.137603082968172\\
1.84	0.13485067338277\\
1.85	0.132153322079294\\
1.86	0.129509927565553\\
1.87	0.126919410390713\\
1.88	0.124380712704042\\
1.89	0.121892797822495\\
1.9	0.119454649806958\\
1.91	0.117065273046993\\
1.92	0.114723691853893\\
1.93	0.112428950061896\\
1.94	0.110180110637387\\
1.95	0.107976255295926\\
1.96	0.105816484126954\\
1.97	0.103699915226012\\
1.98	0.101625684334327\\
1.99	0.0995929444856207\\
2	0.0976008656599932\\
2.01	0.0956486344447337\\
2.02	0.0937354537019301\\
2.03	0.0918605422427345\\
2.04	0.0900231345081515\\
2.05	0.0882224802562205\\
2.06	0.0864578442554627\\
2.07	0.0847285059844664\\
2.08	0.0830337593374898\\
2.09	0.0813729123359575\\
2.1	0.0797452868457358\\
2.11	0.0781502183000684\\
2.12	0.0765870554280602\\
2.13	0.0750551599885981\\
2.14	0.0735539065095992\\
2.15	0.0720826820324799\\
2.16	0.0706408858617417\\
2.17	0.0692279293195698\\
2.18	0.0678432355053461\\
2.19	0.0664862390599761\\
2.2	0.0651563859349345\\
2.21	0.063853133165934\\
2.22	0.0625759486511263\\
2.23	0.0613243109337417\\
2.24	0.0600977089890819\\
2.25	0.0588956420157753\\
2.26	0.0577176192312117\\
2.27	0.0565631596710725\\
2.28	0.0554317919928724\\
2.29	0.0543230542834351\\
2.3	0.0532364938702222\\
2.31	0.0521716671364379\\
2.32	0.051128139339836\\
2.33	0.0501054844351527\\
2.34	0.0491032849000937\\
2.35	0.0481211315648043\\
2.36	0.047158623444753\\
2.37	0.0462153675769591\\
2.38	0.0452909788594991\\
2.39	0.0443850798942244\\
2.4	0.0434973008326277\\
2.41	0.0426272792247931\\
2.42	0.0417746598713706\\
2.43	0.0409390946785115\\
2.44	0.0401202425157088\\
2.45	0.0393177690764809\\
2.46	0.0385313467418442\\
2.47	0.0377606544465182\\
2.48	0.0370053775478075\\
2.49	0.036265207697108\\
2.5	0.0355398427139845\\
2.51	0.0348289864627682\\
2.52	0.0341323487316235\\
2.53	0.0334496451140347\\
2.54	0.0327805968926651\\
2.55	0.0321249309255386\\
2.56	0.0314823795345005\\
2.57	0.0308526803959092\\
2.58	0.0302355764335155\\
2.59	0.0296308157134852\\
2.6	0.0290381513415229\\
2.61	0.0284573413620531\\
2.62	0.0278881486594202\\
2.63	0.0273303408610638\\
2.64	0.0267836902426327\\
2.65	0.0262479736349967\\
2.66	0.0257229723331194\\
2.67	0.0252084720067538\\
2.68	0.0247042626129254\\
2.69	0.0242101383101659\\
2.7	0.0237258973744633\\
2.71	0.0232513421168933\\
2.72	0.0227862788028987\\
2.73	0.0223305175731846\\
2.74	0.0218838723661952\\
2.75	0.0214461608421425\\
2.76	0.0210172043085545\\
2.77	0.0205968276473135\\
2.78	0.0201848592431533\\
2.79	0.0197811309135881\\
2.8	0.0193854778402421\\
2.81	0.018997738501554\\
2.82	0.0186177546068275\\
2.83	0.0182453710316005\\
2.84	0.0178804357543089\\
2.85	0.0175227997942161\\
2.86	0.0171723171505848\\
2.87	0.0168288447430663\\
2.88	0.0164922423532811\\
2.89	0.0161623725675697\\
2.9	0.0158391007208875\\
2.91	0.0155222948418231\\
2.92	0.0152118255987158\\
2.93	0.0149075662468514\\
2.94	0.0146093925767144\\
2.95	0.0143171828632754\\
2.96	0.0140308178162931\\
2.97	0.0137501805316101\\
2.98	0.0134751564434236\\
2.99	0.0132056332775109\\
3	0.01294150100539\\
3.01	0.0126826517993978\\
3.02	0.0124289799886672\\
3.03	0.0121803820159839\\
3.04	0.011936756395507\\
3.05	0.0116980036713357\\
3.06	0.0114640263769038\\
3.07	0.0112347289951873\\
3.08	0.0110100179197082\\
3.09	0.0107898014163173\\
3.1	0.0105739895857429\\
3.11	0.0103624943268876\\
3.12	0.01015522930086\\
3.13	0.00995210989572565\\
3.14	0.00975305319196289\\
3.15	0.00955797792861027\\
3.16	0.00936680447009072\\
3.17	0.00917945477369952\\
3.18	0.00899585235774264\\
3.19	0.00881592227031242\\
3.2	0.00863959105868787\\
3.21	0.00846678673934707\\
3.22	0.00829743876857945\\
3.23	0.00813147801368588\\
3.24	0.00796883672475497\\
3.25	0.00780944850700381\\
3.26	0.00765324829367213\\
3.27	0.00750017231945856\\
3.28	0.00735015809448833\\
3.29	0.00720314437880167\\
3.3	0.00705907115735255\\
3.31	0.0069178796155075\\
3.32	0.00677951211503458\\
3.33	0.00664391217057259\\
3.34	0.00651102442657104\\
3.35	0.00638079463469133\\
3.36	0.00625316963166003\\
3.37	0.00612809731756512\\
3.38	0.00600552663458642\\
3.39	0.00588540754615141\\
3.4	0.00576769101650809\\
3.41	0.00565232899070635\\
3.42	0.00553927437497982\\
3.43	0.00542848101752017\\
3.44	0.00531990368963586\\
3.45	0.00521349806728796\\
3.46	0.00510922071299512\\
3.47	0.00500702905810059\\
3.48	0.00490688138539393\\
3.49	0.00480873681208035\\
3.5	0.00471255527309068\\
3.51	0.00461829750472517\\
3.52	0.00452592502862454\\
3.53	0.00443540013606154\\
3.54	0.00434668587254682\\
3.55	0.00425974602274267\\
3.56	0.00417454509567851\\
3.57	0.00409104831026216\\
3.58	0.00400922158108088\\
3.59	0.00392903150448643\\
3.6	0.00385044534495849\\
3.61	0.00377343102174081\\
3.62	0.00369795709574468\\
3.63	0.00362399275671439\\
3.64	0.00355150781064938\\
3.65	0.00348047266747795\\
3.66	0.00341085832897762\\
3.67	0.00334263637693702\\
3.68	0.0032757789615546\\
3.69	0.00321025879006939\\
3.7	0.0031460491156192\\
3.71	0.00308312372632161\\
3.72	0.00302145693457347\\
3.73	0.0029610235665643\\
3.74	0.0029017989519996\\
3.75	0.00284375891402958\\
3.76	0.00278687975937939\\
3.77	0.00273113826867681\\
3.78	0.00267651168697333\\
3.79	0.00262297771445489\\
3.8	0.00257051449733836\\
3.81	0.00251910061895021\\
3.82	0.0024687150909835\\
3.83	0.00241933734492987\\
3.84	0.00237094722368279\\
3.85	0.00232352497330888\\
3.86	0.00227705123498373\\
3.87	0.00223150703708906\\
3.88	0.00218687378746794\\
3.89	0.00214313326583494\\
3.9	0.00210026761633806\\
3.91	0.00205825934026946\\
3.92	0.00201709128892201\\
3.93	0.00197674665658865\\
3.94	0.00193720897370186\\
3.95	0.00189846210011028\\
3.96	0.0018604902184899\\
3.97	0.00182327782788695\\
3.98	0.00178680973738996\\
3.99	0.00175107105992847\\
4	0.00171604720619566\\
4.01	0.00168172387869265\\
4.02	0.00164808706589181\\
4.03	0.00161512303651694\\
4.04	0.00158281833393774\\
4.05	0.0015511597706765\\
4.06	0.00152013442302458\\
4.07	0.00148972962576663\\
4.08	0.00145993296701033\\
4.09	0.0014307322831195\\
4.1	0.0014021156537486\\
4.11	0.00137407139697654\\
4.12	0.00134658806453781\\
4.13	0.00131965443714901\\
4.14	0.00129325951992883\\
4.15	0.00126739253790971\\
4.16	0.00124204293163917\\
4.17	0.0012172003528692\\
4.18	0.00119285466033184\\
4.19	0.00116899591559926\\
4.2	0.00114561437902665\\
4.21	0.00112270050577624\\
4.22	0.0011002449419209\\
4.23	0.00107823852062557\\
4.24	0.00105667225840521\\
4.25	0.00103553735145744\\
4.26	0.0010148251720687\\
4.27	0.000994527265092134\\
4.28	0.000974635344496026\\
4.29	0.000955141289981249\\
4.3	0.000936037143666347\\
4.31	0.000917315106838946\\
4.32	0.000898967536772131\\
4.33	0.000880986943604504\\
4.34	0.000863365987282643\\
4.35	0.00084609747456472\\
4.36	0.000829174356084049\\
4.37	0.000812589723471371\\
4.38	0.000796336806534692\\
4.39	0.000780408970495536\\
4.4	0.000764799713280473\\
4.41	0.00074950266286682\\
4.42	0.000734511574681436\\
4.43	0.00071982032905154\\
4.44	0.000705422928706523\\
4.45	0.000691313496329723\\
4.46	0.000677486272159168\\
4.47	0.000663935611636311\\
4.48	0.000650655983101793\\
4.49	0.000637641965537295\\
4.5	0.00062488824635255\\
4.51	0.000612389619216631\\
4.52	0.000600140981932601\\
4.53	0.00058813733435469\\
4.54	0.000576373776347115\\
4.55	0.000564845505783741\\
4.56	0.000553547816587741\\
4.57	0.000542476096810475\\
4.58	0.000531625826748785\\
4.59	0.000520992577099955\\
4.6	0.000510572007153569\\
4.61	0.000500359863019541\\
4.62	0.000490351975891579\\
4.63	0.000480544260345386\\
4.64	0.000470932712670905\\
4.65	0.00046151340923791\\
4.66	0.000452282504894298\\
4.67	0.000443236231396416\\
4.68	0.000434370895870775\\
4.69	0.000425682879306543\\
4.7	0.000417168635078186\\
4.71	0.000408824687497658\\
4.72	0.000400647630395548\\
4.73	0.000392634125730609\\
4.74	0.000384780902227098\\
4.75	0.000377084754039369\\
4.76	0.000369542539443182\\
4.77	0.000362151179553175\\
4.78	0.000354907657066006\\
4.79	0.00034780901502862\\
4.8	0.000340852355631157\\
4.81	0.000334034839024008\\
4.82	0.000327353682158522\\
4.83	0.000320806157650911\\
4.84	0.000314389592668871\\
4.85	0.000308101367840477\\
4.86	0.000301938916184898\\
4.87	0.0002958997220645\\
4.88	0.000289981320157913\\
4.89	0.000284181294453631\\
4.9	0.000278497277263747\\
4.91	0.000272926948257411\\
4.92	0.000267468033513624\\
4.93	0.000262118304592973\\
4.94	0.00025687557762794\\
4.95	0.000251737712431393\\
4.96	0.000246702611622926\\
4.97	0.000241768219772656\\
4.98	0.00023693252256216\\
4.99	0.000232193545962184\\
5	0.000227549355426803\\
};
\addlegendentry{\(x_1\)};

\addplot [color=mycolor2,solid]
  table[row sep=crcr]{%
0	0\\
0.01	-0.950069425216098\\
0.02	-1.89479883271611\\
0.03	-2.83280850295291\\
0.04	-3.76212444792955\\
0.05	-4.67978948168419\\
0.06	-5.58110481588912\\
0.07	-6.4580023585369\\
0.08	-7.2950933736677\\
0.09	-8.05850219503889\\
0.1	-8.66077297554567\\
0.11	-8.92697884398766\\
0.12	-8.83472707109159\\
0.13	-8.660262815696\\
0.14	-8.4866931759622\\
0.15	-8.31621529806666\\
0.16	-8.14911906466565\\
0.17	-7.98539087296735\\
0.18	-7.8249700780463\\
0.19	-7.66779007263986\\
0.2	-7.51378460697885\\
0.21	-7.36288869414473\\
0.22	-7.2150387091042\\
0.23	-7.07017237391279\\
0.24	-6.92822872835011\\
0.25	-6.78914809950282\\
0.26	-6.65287207201148\\
0.27	-6.5193434591569\\
0.28	-6.38850627477295\\
0.29	-6.26030570595285\\
0.3	-6.13468808651583\\
0.31	-6.01160087120313\\
0.32	-5.89099261057402\\
0.33	-5.77281292657441\\
0.34	-5.65701248875245\\
0.35	-5.54354299109672\\
0.36	-5.4323571294743\\
0.37	-5.32340857964711\\
0.38	-5.21665197584623\\
0.39	-5.11204288988477\\
0.4	-5.00953781079131\\
0.41	-4.90909412494632\\
0.42	-4.81067009670531\\
0.43	-4.71422484949296\\
0.44	-4.6197183473534\\
0.45	-4.52711137694247\\
0.46	-4.43636552994837\\
0.47	-4.34744318592778\\
0.48	-4.26030749554524\\
0.49	-4.17492236420371\\
0.5	-4.09125243605526\\
0.51	-4.00926307838084\\
0.52	-3.92892036632888\\
0.53	-3.85019106800254\\
0.54	-3.77304262988624\\
0.55	-3.69744316260205\\
0.56	-3.62336142698709\\
0.57	-3.55076682048353\\
0.58	-3.47962936383268\\
0.59	-3.40991968806548\\
0.6	-3.34160902178159\\
0.61	-3.27466917870966\\
0.62	-3.20907254554167\\
0.63	-3.14479207003448\\
0.64	-3.08180124937176\\
0.65	-3.02007411877993\\
0.66	-2.95958524039193\\
0.67	-2.9003096923525\\
0.68	-2.84222305815939\\
0.69	-2.78530141623461\\
0.7	-2.72952132972031\\
0.71	-2.67485983649377\\
0.72	-2.62129443939657\\
0.73	-2.56880309667264\\
0.74	-2.51736421261042\\
0.75	-2.46695662838433\\
0.76	-2.41755961309096\\
0.77	-2.36915285497542\\
0.78	-2.32171645284355\\
0.79	-2.27523090765572\\
0.8	-2.22967711429795\\
0.81	-2.18503635352664\\
0.82	-2.14129028408257\\
0.83	-2.09842093497076\\
0.84	-2.05641069790218\\
0.85	-2.01524231989381\\
0.86	-1.97489889602357\\
0.87	-1.93536386233658\\
0.88	-1.89662098889944\\
0.89	-1.85865437299932\\
0.9	-1.82144843248458\\
0.91	-1.78498789924389\\
0.92	-1.74925781282079\\
0.93	-1.71424351416078\\
0.94	-1.6799306394879\\
0.95	-1.64630511430831\\
0.96	-1.61335314753771\\
0.97	-1.58106122575029\\
0.98	-1.54941610754638\\
0.99	-1.51840481803629\\
1	-1.48801464343792\\
1.01	-1.45823312578559\\
1.02	-1.4290480577478\\
1.03	-1.40044747755161\\
1.04	-1.3724196640112\\
1.05	-1.34495313165871\\
1.06	-1.31803662597484\\
1.07	-1.29165911871736\\
1.08	-1.26580980334531\\
1.09	-1.24047809053695\\
1.1	-1.21565360379948\\
1.11	-1.19132617516848\\
1.12	-1.16748584099538\\
1.13	-1.14412283782097\\
1.14	-1.12122759833316\\
1.15	-1.09879074740729\\
1.16	-1.07680309822717\\
1.17	-1.05525564848534\\
1.18	-1.0341395766606\\
1.19	-1.01344623837153\\
1.2	-0.99316716280415\\
1.21	-0.973294049212399\\
1.22	-0.953818763489712\\
1.23	-0.934733334810378\\
1.24	-0.916029952339164\\
1.25	-0.897700962007788\\
1.26	-0.879738863356886\\
1.27	-0.862136306442093\\
1.28	-0.84488608880293\\
1.29	-0.827981152493186\\
1.3	-0.811414581171546\\
1.31	-0.795179597251203\\
1.32	-0.77926955910725\\
1.33	-0.763677958340661\\
1.34	-0.748398417097702\\
1.35	-0.733424685443618\\
1.36	-0.718750638789502\\
1.37	-0.704370275371239\\
1.38	-0.690277713779458\\
1.39	-0.676467190539458\\
1.4	-0.66293305774007\\
1.41	-0.649669780710463\\
1.42	-0.636671935743907\\
1.43	-0.623934207867538\\
1.44	-0.611451388657181\\
1.45	-0.599218374096307\\
1.46	-0.587230162478232\\
1.47	-0.575481852350674\\
1.48	-0.563968640501786\\
1.49	-0.552685819986854\\
1.5	-0.541628778194793\\
1.51	-0.530792994953664\\
1.52	-0.520174040674383\\
1.53	-0.509767574531886\\
1.54	-0.499569342682945\\
1.55	-0.489575176519923\\
1.56	-0.47978099095972\\
1.57	-0.470182782767199\\
1.58	-0.46077662891239\\
1.59	-0.451558684960793\\
1.6	-0.442525183496099\\
1.61	-0.433672432574674\\
1.62	-0.424996814211168\\
1.63	-0.416494782894606\\
1.64	-0.408162864134358\\
1.65	-0.39999765303536\\
1.66	-0.391995812902025\\
1.67	-0.384154073870232\\
1.68	-0.376469231566842\\
1.69	-0.368938145796181\\
1.7	-0.361557739252937\\
1.71	-0.354324996260952\\
1.72	-0.347236961537366\\
1.73	-0.340290738981614\\
1.74	-0.333483490488768\\
1.75	-0.326812434786736\\
1.76	-0.320274846296827\\
1.77	-0.313868054017224\\
1.78	-0.307589440428888\\
1.79	-0.301436440423447\\
1.8	-0.29540654025263\\
1.81	-0.289497276498803\\
1.82	-0.283706235066181\\
1.83	-0.278031050192312\\
1.84	-0.272469403479405\\
1.85	-0.267019022945118\\
1.86	-0.2616776820924\\
1.87	-0.256443198998017\\
1.88	-0.251313435419374\\
1.89	-0.246286295919266\\
1.9	-0.241359727008204\\
1.91	-0.236531716303954\\
1.92	-0.231800291707942\\
1.93	-0.227163520598195\\
1.94	-0.222619509038471\\
1.95	-0.218166401003267\\
1.96	-0.213802377618367\\
1.97	-0.209525656416639\\
1.98	-0.205334490608753\\
1.99	-0.201227168368539\\
2	-0.197202012132669\\
2.01	-0.193257377914398\\
2.02	-0.189391654631065\\
2.03	-0.185603263445078\\
2.04	-0.181890657118125\\
2.05	-0.178252319378324\\
2.06	-0.174686764300076\\
2.07	-0.171192535696346\\
2.08	-0.167768206523128\\
2.09	-0.164412378295856\\
2.1	-0.161123680517511\\
2.11	-0.157900770118195\\
2.12	-0.154742330905941\\
2.13	-0.151647073028529\\
2.14	-0.148613732446092\\
2.15	-0.145641070414293\\
2.16	-0.142727872977862\\
2.17	-0.13987295047428\\
2.18	-0.137075137047415\\
2.19	-0.134333290170901\\
2.2	-0.13164629018107\\
2.21	-0.129013039819245\\
2.22	-0.1264324637832\\
2.23	-0.123903508287616\\
2.24	-0.121425140633337\\
2.25	-0.118996348785254\\
2.26	-0.116616140958661\\
2.27	-0.114283545213876\\
2.28	-0.111997609059006\\
2.29	-0.109757399060653\\
2.3	-0.107562000462427\\
2.31	-0.105410516811099\\
2.32	-0.103302069590243\\
2.33	-0.101235797861215\\
2.34	-0.0992108579113263\\
2.35	-0.0972264229090591\\
2.36	-0.0952816825661931\\
2.37	-0.0933758428066956\\
2.38	-0.0915081254422438\\
2.39	-0.0896777678542466\\
2.4	-0.0878840226822329\\
2.41	-0.0861261575184816\\
2.42	-0.0844034546087664\\
2.43	-0.0827152105590932\\
2.44	-0.0810607360483105\\
2.45	-0.0794393555464737\\
2.46	-0.0778504070388497\\
2.47	-0.0762932417554475\\
2.48	-0.0747672239059646\\
2.49	-0.0732717304200399\\
2.5	-0.0718061506927082\\
2.51	-0.0703698863349501\\
2.52	-0.0689623509292372\\
2.53	-0.0675829697899706\\
2.54	-0.0662311797287161\\
2.55	-0.0649064288241394\\
2.56	-0.0636081761965467\\
2.57	-0.0623358917869394\\
2.58	-0.0610890561404914\\
2.59	-0.0598671601943613\\
2.6	-0.0586697050697517\\
2.61	-0.0574962018681318\\
2.62	-0.0563461714715384\\
2.63	-0.0552191443468749\\
2.64	-0.0541146603541268\\
2.65	-0.0530322685584174\\
2.66	-0.0519715270458241\\
2.67	-0.0509320027428819\\
2.68	-0.0499132712397003\\
2.69	-0.0489149166166191\\
2.7	-0.0479365312743351\\
2.71	-0.0469777157674281\\
2.72	-0.0460380786412192\\
2.73	-0.0451172362718933\\
2.74	-0.044214812709823\\
2.75	-0.0433304395260269\\
2.76	-0.042463755661701\\
2.77	-0.0416144072807621\\
2.78	-0.0407820476253416\\
2.79	-0.0399663368741716\\
2.8	-0.0391669420038049\\
2.81	-0.0383835366526126\\
2.82	-0.037615800987503\\
2.83	-0.0368634215733079\\
2.84	-0.0361260912447827\\
2.85	-0.0354035089811684\\
2.86	-0.0346953797832626\\
2.87	-0.034001414552952\\
2.88	-0.033321329975154\\
2.89	-0.0326548484021221\\
2.9	-0.0320016977400656\\
2.91	-0.0313616113380381\\
2.92	-0.0307343278790499\\
2.93	-0.0301195912733586\\
2.94	-0.0295171505538952\\
2.95	-0.0289267597737829\\
2.96	-0.0283481779059063\\
2.97	-0.0277811687444898\\
2.98	-0.0272255008086461\\
2.99	-0.0266809472478538\\
3	-0.026147285749327\\
3.01	-0.0256242984472379\\
3.02	-0.0251117718337554\\
3.03	-0.0246094966718646\\
3.04	-0.0241172679099298\\
3.05	-0.0236348845979673\\
3.06	-0.0231621498055933\\
3.07	-0.0226988705416136\\
3.08	-0.0222448576752221\\
3.09	-0.0217999258587756\\
3.1	-0.0213638934521143\\
3.11	-0.0209365824483959\\
3.12	-0.0205178184014139\\
3.13	-0.0201074303543698\\
3.14	-0.0197052507700705\\
3.15	-0.0193111154625221\\
3.16	-0.0189248635298928\\
3.17	-0.0185463372888158\\
3.18	-0.0181753822100082\\
3.19	-0.0178118468551764\\
3.2	-0.017455582815185\\
3.21	-0.0171064446494617\\
3.22	-0.0167642898266153\\
3.23	-0.0164289786662412\\
3.24	-0.0161003742818915\\
3.25	-0.0157783425251855\\
3.26	-0.0154627519310388\\
3.27	-0.0151534736639879\\
3.28	-0.014850381465589\\
3.29	-0.0145533516028683\\
3.3	-0.0142622628178044\\
3.31	-0.0139769962778207\\
3.32	-0.013697435527269\\
3.33	-0.0134234664398825\\
3.34	-0.0131549771721813\\
3.35	-0.0128918581178089\\
3.36	-0.012634001862783\\
3.37	-0.012381303141641\\
3.38	-0.0121336587944632\\
3.39	-0.0118909677247552\\
3.4	-0.0116531308581734\\
3.41	-0.0114200511020761\\
3.42	-0.0111916333058832\\
3.43	-0.0109677842222298\\
3.44	-0.0107484124688958\\
3.45	-0.0105334284914985\\
3.46	-0.0103227445269298\\
3.47	-0.0101162745675263\\
3.48	-0.00991393432595537\\
3.49	-0.00971564120080383\\
3.5	-0.00952131424285493\\
3.51	-0.00933087412204004\\
3.52	-0.00914424309505138\\
3.53	-0.00896134497360261\\
3.54	-0.00878210509332435\\
3.55	-0.00860645028328189\\
3.56	-0.00843430883610272\\
3.57	-0.00826561047870152\\
3.58	-0.00810028634359083\\
3.59	-0.00793826894076561\\
3.6	-0.00777949213015017\\
3.61	-0.00762389109459631\\
3.62	-0.00747140231342152\\
3.63	-0.0073219635364766\\
3.64	-0.00717551375873193\\
3.65	-0.00703199319537214\\
3.66	-0.00689134325738901\\
3.67	-0.00675350652766248\\
3.68	-0.00661842673752024\\
3.69	-0.00648604874376613\\
3.7	-0.00635631850616807\\
3.71	-0.00622918306539629\\
3.72	-0.00610459052140287\\
3.73	-0.00598249001223378\\
3.74	-0.0058628316932647\\
3.75	-0.00574556671685219\\
3.76	-0.00563064721239196\\
3.77	-0.00551802626677596\\
3.78	-0.00540765790524038\\
3.79	-0.00529949707259686\\
3.8	-0.00519349961483896\\
3.81	-0.00508962226111668\\
3.82	-0.00498782260607151\\
3.83	-0.00488805909252478\\
3.84	-0.00479029099451236\\
3.85	-0.00469447840065868\\
3.86	-0.00460058219788326\\
3.87	-0.00450856405543329\\
3.88	-0.00441838640923543\\
3.89	-0.00433001244656076\\
3.9	-0.00424340609099641\\
3.91	-0.00415853198771774\\
3.92	-0.00407535548905524\\
3.93	-0.00399384264035002\\
3.94	-0.00391396016609228\\
3.95	-0.00383567545633701\\
3.96	-0.00375895655339144\\
3.97	-0.00368377213876877\\
3.98	-0.00361009152040284\\
3.99	-0.00353788462011855\\
4	-0.00346712196135286\\
4.01	-0.00339777465712151\\
4.02	-0.00332981439822628\\
4.03	-0.00326321344169826\\
4.04	-0.00319794459947221\\
4.05	-0.00313398122728749\\
4.06	-0.00307129721381098\\
4.07	-0.00300986696997757\\
4.08	-0.00294966541854388\\
4.09	-0.00289066798385088\\
4.1	-0.00283285058179137\\
4.11	-0.00277618960997801\\
4.12	-0.00272066193810814\\
4.13	-0.00266624489852121\\
4.14	-0.00261291627694516\\
4.15	-0.00256065430342786\\
4.16	-0.00250943764344994\\
4.17	-0.00245924538921539\\
4.18	-0.00241005705111635\\
4.19	-0.00236185254936865\\
4.2	-0.00231461220581464\\
4.21	-0.00226831673588996\\
4.22	-0.00222294724075102\\
4.23	-0.00217848519955995\\
4.24	-0.00213491246192384\\
4.25	-0.00209221124048523\\
4.26	-0.00205036410366078\\
4.27	-0.0020093539685252\\
4.28	-0.00196916409383756\\
4.29	-0.00192977807320698\\
4.3	-0.0018911798283951\\
4.31	-0.00185335360275251\\
4.32	-0.00181628395478642\\
4.33	-0.00177995575185706\\
4.34	-0.00174435416400006\\
4.35	-0.00170946465787249\\
4.36	-0.00167527299081996\\
4.37	-0.00164176520506234\\
4.38	-0.00160892762199588\\
4.39	-0.0015767468366092\\
4.4	-0.00154520971201098\\
4.41	-0.00151430337406722\\
4.42	-0.00148401520614557\\
4.43	-0.00145433284396497\\
4.44	-0.00142524417054821\\
4.45	-0.00139673731127544\\
4.46	-0.00136880062903665\\
4.47	-0.00134142271948113\\
4.48	-0.0013145924063619\\
4.49	-0.00128829873697326\\
4.5	-0.00126253097767967\\
4.51	-0.00123727860953395\\
4.52	-0.00121253132398322\\
4.53	-0.00118827901866069\\
4.54	-0.0011645117932616\\
4.55	-0.00114121994550166\\
4.56	-0.00111839396715638\\
4.57	-0.00109602454017955\\
4.58	-0.00107410253289938\\
4.59	-0.00105261899629074\\
4.6	-0.00103156516032194\\
4.61	-0.0010109324303746\\
4.62	-0.000990712383735117\\
4.63	-0.00097089676615629\\
4.64	-0.000951477488487767\\
4.65	-0.000932446623373842\\
4.66	-0.000913796402017322\\
4.67	-0.000895519211008121\\
4.68	-0.000877607589215283\\
4.69	-0.000860054224741174\\
4.7	-0.000842851951936591\\
4.71	-0.000825993748475573\\
4.72	-0.00080947273248873\\
4.73	-0.000793282159753896\\
4.74	-0.000777415420942983\\
4.75	-0.000761866038923899\\
4.76	-0.000746627666116435\\
4.77	-0.000731694081901033\\
4.78	-0.000717059190079385\\
4.79	-0.000702717016385827\\
4.8	-0.000688661706048509\\
4.81	-0.000674887521399335\\
4.82	-0.000661388839531727\\
4.83	-0.000648160150005215\\
4.84	-0.00063519605259596\\
4.85	-0.000622491255092256\\
4.86	-0.000610040571134127\\
4.87	-0.000597838918096144\\
4.88	-0.000585881315012578\\
4.89	-0.000574162880544061\\
4.9	-0.000562678830984911\\
4.91	-0.000551424478310321\\
4.92	-0.000540395228262597\\
4.93	-0.000529586578475686\\
4.94	-0.000518994116637206\\
4.95	-0.000508613518687242\\
4.96	-0.000498440547053173\\
4.97	-0.000488471048919798\\
4.98	-0.000478700954534067\\
4.99	-0.000469126275543715\\
5	-0.000459743103369137\\
};
\addlegendentry{\(x_2\)};

\addplot [color=mycolor3,solid]
  table[row sep=crcr]{%
0	10\\
0.01	9.04042184651033\\
0.02	8.06723378014244\\
0.03	7.08193534353757\\
0.04	6.08665342185972\\
0.05	5.08454655233347\\
0.06	4.08058971438661\\
0.07	3.08325105187327\\
0.08	2.10854441004356\\
0.09	1.19143654068218\\
0.1	0.421629401884791\\
0.11	-0.0204855284060343\\
0.12	-0.105624459927217\\
0.13	-0.106100307065658\\
0.14	-0.103993783817895\\
0.15	-0.101539184275508\\
0.16	-0.0990906889460437\\
0.17	-0.0967021182190662\\
0.18	-0.0943795678505888\\
0.19	-0.0921219083838398\\
0.2	-0.0899270408069563\\
0.21	-0.0877928168584781\\
0.22	-0.0857171641235137\\
0.23	-0.0836980982996902\\
0.24	-0.0817337204757802\\
0.25	-0.0798222127631396\\
0.26	-0.0779618339898178\\
0.27	-0.0761509156476006\\
0.28	-0.074387858095549\\
0.29	-0.0726711270032521\\
0.3	-0.070999250016281\\
0.31	-0.0693708136275681\\
0.32	-0.0677844602398032\\
0.33	-0.066238885405153\\
0.34	-0.0647328352297487\\
0.35	-0.0632651039314283\\
0.36	-0.0618345315400983\\
0.37	-0.0604400017309628\\
0.38	-0.059080439781618\\
0.39	-0.0577548106447168\\
0.4	-0.0564621171285342\\
0.41	-0.0552013981783634\\
0.42	-0.0539717272521729\\
0.43	-0.0527722107845037\\
0.44	-0.0516019867329494\\
0.45	-0.0504602232020437\\
0.46	-0.0493461171397227\\
0.47	-0.0482588931018535\\
0.48	-0.0471978020807109\\
0.49	-0.0461621203934666\\
0.5	-0.0451511486271396\\
0.51	-0.0441642106366165\\
0.52	-0.0432006525926312\\
0.53	-0.0422598420767648\\
0.54	-0.0413411672207666\\
0.55	-0.0404440358876412\\
0.56	-0.0395678748921213\\
0.57	-0.0387121292583168\\
0.58	-0.0378762615124586\\
0.59	-0.0370597510087811\\
0.6	-0.0362620932867448\\
0.61	-0.035482799457867\\
0.62	-0.0347213956205783\\
0.63	-0.0339774223015921\\
0.64	-0.0332504339223849\\
0.65	-0.0325399982894616\\
0.66	-0.0318456961071543\\
0.67	-0.0311671205118018\\
0.68	-0.030503876626184\\
0.69	-0.02985558113321\\
0.7	-0.029221861867851\\
0.71	-0.0286023574264203\\
0.72	-0.0279967167923338\\
0.73	-0.0274045989775207\\
0.74	-0.0268256726787359\\
0.75	-0.0262596159480273\\
0.76	-0.0257061158766883\\
0.77	-0.0251648682920345\\
0.78	-0.0246355774664031\\
0.79	-0.0241179558377889\\
0.8	-0.0236117237415745\\
0.81	-0.0231166091528299\\
0.82	-0.0226323474387033\\
0.83	-0.0221586811204255\\
0.84	-0.0216953596444962\\
0.85	-0.021242139162635\\
0.86	-0.0207987823200981\\
0.87	-0.020365058051993\\
0.88	-0.0199407413872279\\
0.89	-0.019525613259763\\
0.9	-0.0191194603268461\\
0.91	-0.0187220747939207\\
0.92	-0.0183332542459267\\
0.93	-0.017952801484713\\
0.94	-0.0175805243722997\\
0.95	-0.0172162356797529\\
0.96	-0.0168597529414201\\
0.97	-0.0165108983143136\\
0.98	-0.0161694984424234\\
0.99	-0.015835384325755\\
1	-0.0155083911939036\\
1.01	-0.0151883583839687\\
1.02	-0.0148751292226488\\
1.03	-0.014568550912333\\
1.04	-0.0142684744210391\\
1.05	-0.0139747543760418\\
1.06	-0.0136872489610445\\
1.07	-0.0134058198167577\\
1.08	-0.0131303319447527\\
1.09	-0.0128606536144547\\
1.1	-0.0125966562731668\\
1.11	-0.0123382144590007\\
1.12	-0.0120852057166041\\
1.13	-0.0118375105155821\\
1.14	-0.0115950121715096\\
1.15	-0.0113575967694366\\
1.16	-0.0111251530897953\\
1.17	-0.0108975725366223\\
1.18	-0.010674749068005\\
1.19	-0.0104565791286779\\
1.2	-0.0102429615846846\\
1.21	-0.0100337976600376\\
1.22	-0.00982899087529732\\
1.23	-0.00962844698800625\\
1.24	-0.00943207393491408\\
1.25	-0.009239781775923\\
1.26	-0.00905148263970379\\
1.27	-0.00886709067091374\\
1.28	-0.00868652197897046\\
1.29	-0.00850969458831863\\
1.3	-0.00833652839014865\\
1.31	-0.00816694509550864\\
1.32	-0.00800086818976797\\
1.33	-0.00783822288838543\\
1.34	-0.00767893609393877\\
1.35	-0.00752293635437196\\
1.36	-0.00737015382242112\\
1.37	-0.0072205202161828\\
1.38	-0.00707396878078037\\
1.39	-0.00693043425110063\\
1.4	-0.00678985281556033\\
1.41	-0.00665216208087249\\
1.42	-0.00651730103777948\\
1.43	-0.00638521002771975\\
1.44	-0.00625583071040514\\
1.45	-0.0061291060322709\\
1.46	-0.00600498019577966\\
1.47	-0.00588339862954701\\
1.48	-0.00576430795926497\\
1.49	-0.00564765597940042\\
1.5	-0.00553339162564093\\
1.51	-0.0054214649480695\\
1.52	-0.00531182708504063\\
1.53	-0.0052044302377442\\
1.54	-0.00509922764542708\\
1.55	-0.00499617356126125\\
1.56	-0.00489522322883351\\
1.57	-0.00479633285924075\\
1.58	-0.00469945960877299\\
1.59	-0.00460456155716549\\
1.6	-0.0045115976864053\\
1.61	-0.0044205278600733\\
1.62	-0.00433131280321003\\
1.63	-0.00424391408268648\\
1.64	-0.00415829408806739\\
1.65	-0.00407441601295305\\
1.66	-0.00399224383678493\\
1.67	-0.00391174230710251\\
1.68	-0.00383287692223994\\
1.69	-0.00375561391444679\\
1.7	-0.00367992023342506\\
1.71	-0.00360576353026865\\
1.72	-0.00353311214179364\\
1.73	-0.00346193507525056\\
1.74	-0.00339220199340662\\
1.75	-0.00332388319998816\\
1.76	-0.00325694962547435\\
1.77	-0.00319137281323112\\
1.78	-0.00312712490597755\\
1.79	-0.0030641786325748\\
1.8	-0.00300250729512952\\
1.81	-0.00294208475640395\\
1.82	-0.0028828854275223\\
1.83	-0.00282488425596844\\
1.84	-0.00276805671386537\\
1.85	-0.00271237878652991\\
1.86	-0.00265782696129485\\
1.87	-0.00260437821659204\\
1.88	-0.00255201001128955\\
1.89	-0.00250070027427624\\
1.9	-0.00245042739428797\\
1.91	-0.00240117020996772\\
1.92	-0.00235290800015578\\
1.93	-0.00230562047440222\\
1.94	-0.00225928776369747\\
1.95	-0.00221389041141454\\
1.96	-0.00216940936445809\\
1.97	-0.00212582596461469\\
1.98	-0.0020831219401001\\
1.99	-0.00204127939729726\\
2	-0.00200028081268219\\
2.01	-0.00196010902493082\\
2.02	-0.00192074722720476\\
2.03	-0.00188217895960932\\
2.04	-0.00184438810182158\\
2.05	-0.00180735886588251\\
2.06	-0.00177107578915076\\
2.07	-0.00173552372741315\\
2.08	-0.0017006878481485\\
2.09	-0.00166655362394086\\
2.1	-0.00163310682603904\\
2.11	-0.00160033351805811\\
2.12	-0.00156822004982055\\
2.13	-0.0015367530513325\\
2.14	-0.0015059194268931\\
2.15	-0.00147570634933281\\
2.16	-0.0014461012543783\\
2.17	-0.00141709183514041\\
2.18	-0.00138866603672305\\
2.19	-0.00136081205094901\\
2.2	-0.00133351831120135\\
2.21	-0.00130677348737657\\
2.22	-0.0012805664809474\\
2.23	-0.00125488642013287\\
2.24	-0.00122972265517267\\
2.25	-0.00120506475370381\\
2.26	-0.00118090249623715\\
2.27	-0.00115722587173121\\
2.28	-0.00113402507326152\\
2.29	-0.00111129049378288\\
2.3	-0.00108901272198278\\
2.31	-0.00106718253822345\\
2.32	-0.00104579091057112\\
2.33	-0.00102482899090987\\
2.34	-0.00100428811113891\\
2.35	-0.000984159779450391\\
2.36	-0.000964435676687123\\
2.37	-0.000945107652777374\\
2.38	-0.000926167723245691\\
2.39	-0.000907608065797746\\
2.4	-0.000889421016977476\\
2.41	-0.000871599068895279\\
2.42	-0.000854134866025258\\
2.43	-0.000837021202070173\\
2.44	-0.00082025101689287\\
2.45	-0.000803817393512013\\
2.46	-0.000787713555161385\\
2.47	-0.000771932862411201\\
2.48	-0.000756468810349645\\
2.49	-0.000741315025823919\\
2.5	-0.000726465264739101\\
2.51	-0.000711913409413636\\
2.52	-0.000697653465990239\\
2.53	-0.000683679561901102\\
2.54	-0.000669985943385965\\
2.55	-0.000656566973062281\\
2.56	-0.000643417127545759\\
2.57	-0.000630530995121052\\
2.58	-0.000617903273460507\\
2.59	-0.000605528767390805\\
2.6	-0.000593402386705903\\
2.61	-0.000581519144025502\\
2.62	-0.000569874152698061\\
2.63	-0.000558462624747333\\
2.64	-0.000547279868861474\\
2.65	-0.000536321288423985\\
2.66	-0.000525582379585297\\
2.67	-0.000515058729374414\\
2.68	-0.000504746013849589\\
2.69	-0.000494639996287273\\
2.7	-0.000484736525408438\\
2.71	-0.000475031533641604\\
2.72	-0.000465521035421738\\
2.73	-0.000456201125524108\\
2.74	-0.000447067977432679\\
2.75	-0.000438117841741983\\
2.76	-0.000429347044591993\\
2.77	-0.000420751986135157\\
2.78	-0.00041232913903496\\
2.79	-0.000404075046995418\\
2.8	-0.000395986323320732\\
2.81	-0.000388059649504537\\
2.82	-0.000380291773848088\\
2.83	-0.000372679510106902\\
2.84	-0.000365219736164941\\
2.85	-0.000357909392736264\\
2.86	-0.000350745482092955\\
2.87	-0.000343725066819386\\
2.88	-0.000336845268591741\\
2.89	-0.000330103266982719\\
2.9	-0.000323496298290524\\
2.91	-0.000317021654391847\\
2.92	-0.000310676681618317\\
2.93	-0.000304458779655837\\
2.94	-0.000298365400466426\\
2.95	-0.00029239404723207\\
2.96	-0.00028654227332009\\
2.97	-0.000280807681269618\\
2.98	-0.000275187921798769\\
2.99	-0.000269680692831942\\
3	-0.000264283738547085\\
3.01	-0.000258994848442193\\
3.02	-0.000253811856420962\\
3.03	-0.000248732639896897\\
3.04	-0.000243755118915785\\
3.05	-0.000238877255295904\\
3.06	-0.000234097051785762\\
3.07	-0.000229412551238945\\
3.08	-0.000224821835805702\\
3.09	-0.00022032302614098\\
3.1	-0.000215914280628528\\
3.11	-0.000211593794620719\\
3.12	-0.00020735979969385\\
3.13	-0.000203210562918488\\
3.14	-0.000199144386144684\\
3.15	-0.000195159605301577\\
3.16	-0.000191254589711309\\
3.17	-0.000187427741416751\\
3.18	-0.000183677494522898\\
3.19	-0.000180002314551578\\
3.2	-0.000176400697809245\\
3.21	-0.00017287117076752\\
3.22	-0.00016941228945638\\
3.23	-0.00016602263886947\\
3.24	-0.000162700832381592\\
3.25	-0.000159445511177863\\
3.26	-0.000156255343694494\\
3.27	-0.000153129025070791\\
3.28	-0.000150065276612349\\
3.29	-0.00014706284526496\\
3.3	-0.000144120503099259\\
3.31	-0.000141237046805702\\
3.32	-0.000138411297199803\\
3.33	-0.000135642098737303\\
3.34	-0.000132928319039178\\
3.35	-0.0001302688484262\\
3.36	-0.000127662599462907\\
3.37	-0.000125108506510766\\
3.38	-0.000122605525290349\\
3.39	-0.000120152632452349\\
3.4	-0.000117748825157239\\
3.41	-0.000115393120663405\\
3.42	-0.000113084555923579\\
3.43	-0.000110822187189439\\
3.44	-0.000108605089624113\\
3.45	-0.000106432356922549\\
3.46	-0.000104303100939524\\
3.47	-0.000102216451325103\\
3.48	-0.000100171555167509\\
3.49	-9.81675766431253e-05\\
3.5	-9.62036966735743e-05\\
3.51	-9.42791125897006e-05\\
3.52	-9.23930378023113e-05\\
3.53	-9.05447014795391e-05\\
3.54	-8.87333482307087e-05\\
3.55	-8.69582377965583e-05\\
3.56	-8.5218644745708e-05\\
3.57	-8.35138581771998e-05\\
3.58	-8.18431814290747e-05\\
3.59	-8.0205931792747e-05\\
3.6	-7.86014402331904e-05\\
3.61	-7.70290511146899e-05\\
3.62	-7.54881219321582e-05\\
3.63	-7.39780230478125e-05\\
3.64	-7.24981374331767e-05\\
3.65	-7.1047860416249e-05\\
3.66	-6.96265994337697e-05\\
3.67	-6.82337737884352e-05\\
3.68	-6.68688144110389e-05\\
3.69	-6.55311636273449e-05\\
3.7	-6.42202749296722e-05\\
3.71	-6.29356127530628e-05\\
3.72	-6.16766522559403e-05\\
3.73	-6.04428791051784e-05\\
3.74	-5.92337892654895e-05\\
3.75	-5.80488887930346e-05\\
3.76	-5.68876936331841e-05\\
3.77	-5.57497294223384e-05\\
3.78	-5.46345312937204e-05\\
3.79	-5.35416436870896e-05\\
3.8	-5.24706201622356e-05\\
3.81	-5.14210232162617e-05\\
3.82	-5.03924241045031e-05\\
3.83	-4.93844026650484e-05\\
3.84	-4.83965471467906e-05\\
3.85	-4.74284540409224e-05\\
3.86	-4.64797279158043e-05\\
3.87	-4.55499812551715e-05\\
3.88	-4.46388342995492e-05\\
3.89	-4.374591489089e-05\\
3.9	-4.28708583202937e-05\\
3.91	-4.20133071788106e-05\\
3.92	-4.11729112112132e-05\\
3.93	-4.03493271727163e-05\\
3.94	-3.95422186885647e-05\\
3.95	-3.87512561164437e-05\\
3.96	-3.79761164116298e-05\\
3.97	-3.72164829948781e-05\\
3.98	-3.64720456229256e-05\\
3.99	-3.57425002616121e-05\\
4	-3.50275489615385e-05\\
4.01	-3.43268997362151e-05\\
4.02	-3.36402664426606e-05\\
4.03	-3.29673686643868e-05\\
4.04	-3.23079315967326e-05\\
4.05	-3.16616859344981e-05\\
4.06	-3.10283677618299e-05\\
4.07	-3.04077184443089e-05\\
4.08	-2.97994845232107e-05\\
4.09	-2.92034176118754e-05\\
4.1	-2.861927429416e-05\\
4.11	-2.80468160249259e-05\\
4.12	-2.74858090325179e-05\\
4.13	-2.69360242231972e-05\\
4.14	-2.63972370874942e-05\\
4.15	-2.58692276084325e-05\\
4.16	-2.53517801715921e-05\\
4.17	-2.4844683476982e-05\\
4.18	-2.43477304526681e-05\\
4.19	-2.38607181701337e-05\\
4.2	-2.33834477613463e-05\\
4.21	-2.2915724337472e-05\\
4.22	-2.2457356909224e-05\\
4.23	-2.20081583088083e-05\\
4.24	-2.15679451134329e-05\\
4.25	-2.11365375703452e-05\\
4.26	-2.07137595233707e-05\\
4.27	-2.02994383409357e-05\\
4.28	-1.98934048455082e-05\\
4.29	-1.94954932444789e-05\\
4.3	-1.91055410624025e-05\\
4.31	-1.87233890746137e-05\\
4.32	-1.83488812421613e-05\\
4.33	-1.79818646480539e-05\\
4.34	-1.76221894347746e-05\\
4.35	-1.72697087430524e-05\\
4.36	-1.69242786518579e-05\\
4.37	-1.65857581195998e-05\\
4.38	-1.62540089265009e-05\\
4.39	-1.59288956181254e-05\\
4.4	-1.56102854500389e-05\\
4.41	-1.52980483335771e-05\\
4.42	-1.49920567826965e-05\\
4.43	-1.4692185861894e-05\\
4.44	-1.43983131351661e-05\\
4.45	-1.41103186159918e-05\\
4.46	-1.38280847183159e-05\\
4.47	-1.35514962085114e-05\\
4.48	-1.32804401583109e-05\\
4.49	-1.30148058986703e-05\\
4.5	-1.27544849745649e-05\\
4.51	-1.24993711006847e-05\\
4.52	-1.22493601180198e-05\\
4.53	-1.20043499513131e-05\\
4.54	-1.17642405673651e-05\\
4.55	-1.15289339341752e-05\\
4.56	-1.12983339808975e-05\\
4.57	-1.10723465585995e-05\\
4.58	-1.08508794018102e-05\\
4.59	-1.06338420908304e-05\\
4.6	-1.04211460148027e-05\\
4.61	-1.02127043355206e-05\\
4.62	-1.00084319519589e-05\\
4.63	-9.80824546551731e-06\\
4.64	-9.61206314595753e-06\\
4.65	-9.41980489802235e-06\\
4.66	-9.23139222872482e-06\\
4.67	-9.04674821528896e-06\\
4.68	-8.86579747373423e-06\\
4.69	-8.68846612808879e-06\\
4.7	-8.51468178021805e-06\\
4.71	-8.34437348025689e-06\\
4.72	-8.17747169763435e-06\\
4.73	-8.01390829267856e-06\\
4.74	-7.85361648878726e-06\\
4.75	-7.69653084516043e-06\\
4.76	-7.54258723007211e-06\\
4.77	-7.39172279468276e-06\\
4.78	-7.24387594737213e-06\\
4.79	-7.09898632858736e-06\\
4.8	-6.95699478619365e-06\\
4.81	-6.81784335131944e-06\\
4.82	-6.68147521468274e-06\\
4.83	-6.54783470339329e-06\\
4.84	-6.41686725821801e-06\\
4.85	-6.28851941130142e-06\\
4.86	-6.16273876433149e-06\\
4.87	-6.03947396714385e-06\\
4.88	-5.91867469675197e-06\\
4.89	-5.80029163679829e-06\\
4.9	-5.68427645741723e-06\\
4.91	-5.57058179549903e-06\\
4.92	-5.4591612353497e-06\\
4.93	-5.3499692897396e-06\\
4.94	-5.24296138132614e-06\\
4.95	-5.13809382445571e-06\\
4.96	-5.03532380732158e-06\\
4.97	-4.93460937448599e-06\\
4.98	-4.8359094097462e-06\\
4.99	-4.73918361934783e-06\\
5	-4.64439251553103e-06\\
};
\addlegendentry{\(s(\bm x)\)};

\end{axis}
\end{tikzpicture}%
    \caption{Zustände über die Zeit.}
    \label{fig:timeseries_states}
\end{subfigure}

\begin{subfigure}[b]{0.95\textwidth}
    % This file was created by matlab2tikz.
%
\definecolor{mycolor1}{rgb}{0.00000,0.56863,0.86275}%
%
\begin{tikzpicture}

\begin{axis}[%
width=0.951\figurewidth,
height=\figureheight,
at={(0\figurewidth,0\figureheight)},
scale only axis,
xmin=0,
xmax=5,
xlabel={Time \(t\) [s]},
ymin=-100,
ymax=20,
ylabel={Control input},
axis background/.style={fill=white},
axis x line*=bottom,
axis y line*=left,
legend style={legend cell align=left,align=left,draw=white!15!black},
legend pos = south east
]
\addplot [color=mycolor1,solid]
  table[row sep=crcr]{%
0	-95.2380952380952\\
0.01	-94.7591415972566\\
0.02	-94.1638104803569\\
0.03	-93.4053776859734\\
0.04	-92.4088916180013\\
0.05	-91.0467216037249\\
0.06	-89.0843749129154\\
0.07	-86.0461912168111\\
0.08	-80.8322220593649\\
0.09	-70.4393284658293\\
0.1	-45.7482585757933\\
0.11	3.93584975720083\\
0.12	17.4405868514476\\
0.13	17.5054039453183\\
0.14	17.2176910763124\\
0.15	16.8798952636479\\
0.16	16.540181774544\\
0.17	16.2060960178399\\
0.18	15.8786696171079\\
0.19	15.5579293857376\\
0.2	15.2437563607774\\
0.21	14.9360139049838\\
0.22	14.6345658577009\\
0.23	14.339278908652\\
0.24	14.0500228195355\\
0.25	13.7666703699997\\
0.26	13.4890972733662\\
0.27	13.2171820923006\\
0.28	12.95080615774\\
0.29	12.6898534912237\\
0.3	12.4342107304443\\
0.31	12.1837670578149\\
0.32	11.9384141318652\\
0.33	11.6980460212933\\
0.34	11.4625591415122\\
0.35	11.2318521935508\\
0.36	11.0058261051698\\
0.37	10.784383974072\\
0.38	10.5674310130927\\
0.39	10.3548744972647\\
0.4	10.1466237126601\\
0.41	9.9425899069205\\
0.42	9.74268624138728\\
0.43	9.54682774476099\\
0.44	9.3549312682102\\
0.45	9.166915441867\\
0.46	8.98270063264538\\
0.47	8.80220890331972\\
0.48	8.62536397281605\\
0.49	8.45209117765443\\
0.5	8.28231743450313\\
0.51	8.11597120379322\\
0.52	7.95298245435451\\
0.53	7.7932826290283\\
0.54	7.6368046112235\\
0.55	7.48348269237808\\
0.56	7.33325254029112\\
0.57	7.18605116829549\\
0.58	7.04181690524026\\
0.59	6.90048936625213\\
0.6	6.76200942425276\\
0.61	6.62631918220165\\
0.62	6.4933619460433\\
0.63	6.36308219833338\\
0.64	6.23542557252275\\
0.65	6.11033882787797\\
0.66	5.98776982501671\\
0.67	5.86766750204173\\
0.68	5.74998185125014\\
0.69	5.6346638964069\\
0.7	5.52166567055914\\
0.71	5.41094019437884\\
0.72	5.3024414550186\\
0.73	5.19612438546231\\
0.74	5.09194484436117\\
0.75	4.9898595963367\\
0.76	4.88982629274147\\
0.77	4.79180345286173\\
0.78	4.69575044555203\\
0.79	4.60162747128877\\
0.8	4.50939554463221\\
0.81	4.41901647708461\\
0.82	4.33045286033654\\
0.83	4.24366804988828\\
0.84	4.15862614903845\\
0.85	4.07529199322987\\
0.86	3.99363113474305\\
0.87	3.91360982772947\\
0.88	3.83519501357501\\
0.89	3.75835430658549\\
0.9	3.68305597998737\\
0.91	3.60926895223395\\
0.92	3.53696277361135\\
0.93	3.46610761313603\\
0.94	3.39667424573609\\
0.95	3.32863403971193\\
0.96	3.26195894446649\\
0.97	3.19662147850096\\
0.98	3.13259471766852\\
0.99	3.06985228368026\\
1	3.00836833285808\\
1.01	2.94811754512684\\
1.02	2.8890751132429\\
1.03	2.8312167322513\\
1.04	2.77451858916735\\
1.05	2.71895735287756\\
1.06	2.66451016425413\\
1.07	2.611154626479\\
1.08	2.55886879557266\\
1.09	2.50763117112172\\
1.1	2.45742068720282\\
1.11	2.40821670349716\\
1.12	2.35999899659125\\
1.13	2.31274775146081\\
1.14	2.26644355313269\\
1.15	2.22106737852133\\
1.16	2.1766005884357\\
1.17	2.13302491975358\\
1.18	2.09032247775844\\
1.19	2.04847572863626\\
1.2	2.00746749212818\\
1.21	1.96728093433637\\
1.22	1.92789956067866\\
1.23	1.88930720898962\\
1.24	1.85148804276488\\
1.25	1.81442654454453\\
1.26	1.77810750943441\\
1.27	1.74251603876033\\
1.28	1.70763753385421\\
1.29	1.67345768996753\\
1.3	1.63996249031121\\
1.31	1.60713820021758\\
1.32	1.57497136142279\\
1.33	1.54344878646682\\
1.34	1.51255755320877\\
1.35	1.48228499945452\\
1.36	1.45261871769475\\
1.37	1.42354654995136\\
1.38	1.39505658272878\\
1.39	1.36713714206943\\
1.4	1.33977678870998\\
1.41	1.31296431333706\\
1.42	1.28668873193995\\
1.43	1.26093928125788\\
1.44	1.23570541432118\\
1.45	1.21097679608256\\
1.46	1.18674329913833\\
1.47	1.16299499953652\\
1.48	1.13972217267044\\
1.49	1.11691528925638\\
1.5	1.09456501139267\\
1.51	1.07266218869959\\
1.52	1.05119785453719\\
1.53	1.03016322230093\\
1.54	1.00954968179177\\
1.55	0.989348795660759\\
1.56	0.969552295925535\\
1.57	0.950152080557641\\
1.58	0.931140210139291\\
1.59	0.912508904587825\\
1.6	0.894250539946878\\
1.61	0.876357645242296\\
1.62	0.858822899402264\\
1.63	0.841639128239545\\
1.64	0.824799301494981\\
1.65	0.808296529940998\\
1.66	0.792124062543667\\
1.67	0.776275283682226\\
1.68	0.760743710425132\\
1.69	0.745522989860835\\
1.7	0.730606896482917\\
1.71	0.715989329628061\\
1.72	0.701664310965655\\
1.73	0.687625982038367\\
1.74	0.673868601852328\\
1.75	0.660386544516003\\
1.76	0.647174296926884\\
1.77	0.634226456504781\\
1.78	0.621537728970971\\
1.79	0.609102926172129\\
1.8	0.596916963948222\\
1.81	0.584974860043563\\
1.82	0.573271732059729\\
1.83	0.561802795450036\\
1.84	0.550563361554357\\
1.85	0.539548835673626\\
1.86	0.528754715183119\\
1.87	0.518176587683785\\
1.88	0.507810129190851\\
1.89	0.497651102358921\\
1.9	0.487695354742935\\
1.91	0.477938817093957\\
1.92	0.468377501689519\\
1.93	0.45900750069742\\
1.94	0.449824984572593\\
1.95	0.440826200486194\\
1.96	0.432007470786338\\
1.97	0.423365191489772\\
1.98	0.414895830804013\\
1.99	0.406595927679059\\
2	0.398462090388468\\
2.01	0.390490995138713\\
2.02	0.382679384706783\\
2.03	0.37502406710496\\
2.04	0.36752191427263\\
2.05	0.360169860794242\\
2.06	0.352964902643169\\
2.07	0.345904095950765\\
2.08	0.338984555800182\\
2.09	0.332203455044392\\
2.1	0.325558023148058\\
2.11	0.319045545052552\\
2.12	0.312663360063916\\
2.13	0.306408860763033\\
2.14	0.300279491937806\\
2.15	0.294272749536707\\
2.16	0.288386179643405\\
2.17	0.282617377471914\\
2.18	0.276963986382042\\
2.19	0.271423696914454\\
2.2	0.265994245845252\\
2.21	0.260673415259463\\
2.22	0.255459031643126\\
2.23	0.250348964993644\\
2.24	0.245341127947967\\
2.25	0.240433474928269\\
2.26	0.235624001304801\\
2.27	0.230910742575504\\
2.28	0.226291773562119\\
2.29	0.22176520762241\\
2.3	0.217329195878217\\
2.31	0.212981926458942\\
2.32	0.208721623760288\\
2.33	0.204546547717791\\
2.34	0.200454993095014\\
2.35	0.196445288785907\\
2.36	0.192515797131266\\
2.37	0.188664913248831\\
2.38	0.184891064376853\\
2.39	0.181192709230826\\
2.4	0.177568337373077\\
2.41	0.174016468595056\\
2.42	0.170535652311972\\
2.43	0.167124466969558\\
2.44	0.163781519462799\\
2.45	0.160505444566211\\
2.46	0.157294904375608\\
2.47	0.154148587761066\\
2.48	0.151065209830797\\
2.49	0.148043511405822\\
2.5	0.145082258505152\\
2.51	0.142180241841286\\
2.52	0.139336276325814\\
2.53	0.136549200584953\\
2.54	0.133817876484756\\
2.55	0.131141188665884\\
2.56	0.128518044087622\\
2.57	0.125947371581139\\
2.58	0.123428121411587\\
2.59	0.120959264849064\\
2.6	0.118539793748121\\
2.61	0.116168720135709\\
2.62	0.113845075807383\\
2.63	0.111567911931597\\
2.64	0.109336298661908\\
2.65	0.107149324756982\\
2.66	0.105006097208176\\
2.67	0.102905740874602\\
2.68	0.100847398125496\\
2.69	0.0988302284897481\\
2.7	0.096853408312449\\
2.71	0.0949161304183209\\
2.72	0.0930176037818977\\
2.73	0.0911570532042792\\
2.74	0.0893337189963993\\
2.75	0.0875468566686068\\
2.76	0.0857957366264803\\
2.77	0.0840796438727254\\
2.78	0.0823978777150389\\
2.79	0.0807497514798355\\
2.8	0.0791345922316958\\
2.81	0.0775517404984347\\
2.82	0.0760005500016704\\
2.83	0.0744803873928081\\
2.84	0.0729906319942694\\
2.85	0.0715306755459594\\
2.86	0.0700999219567483\\
2.87	0.0686977870609815\\
2.88	0.0673236983798216\\
2.89	0.0659770948874071\\
2.9	0.064657426781663\\
2.91	0.0633641552597102\\
2.92	0.062096752297777\\
2.93	0.0608547004355056\\
2.94	0.0596374925645815\\
2.95	0.0584446317215979\\
2.96	0.0572756308850586\\
2.97	0.0561300127764488\\
2.98	0.0550073096652927\\
2.99	0.0539070631780957\\
3	0.0528288241111388\\
3.01	0.0517721522469892\\
3.02	0.0507366161747127\\
3.03	0.0497217931136563\\
3.04	0.0487272687407843\\
3.05	0.0477526370214512\\
3.06	0.0467975000435702\\
3.07	0.0458614678551006\\
3.08	0.0449441583047829\\
3.09	0.0440451968860661\\
3.1	0.0431642165841604\\
3.11	0.0423008577261399\\
3.12	0.0414547678340612\\
3.13	0.040625601481006\\
3.14	0.0398130201500201\\
3.15	0.0390166920958561\\
3.16	0.0382362922095046\\
3.17	0.0374715018854185\\
3.18	0.0367220088914057\\
3.19	0.0359875072411189\\
3.2	0.0352676970691027\\
3.21	0.0345622845083292\\
3.22	0.0338709815702081\\
3.23	0.0331935060269661\\
3.24	0.0325295812964106\\
3.25	0.0318789363289751\\
3.26	0.031241305497043\\
3.27	0.0306164284864676\\
3.28	0.0300040501902872\\
3.29	0.02940392060454\\
3.3	0.0288157947261856\\
3.31	0.0282394324530542\\
3.32	0.0276745984858087\\
3.33	0.0271210622318584\\
3.34	0.0265785977112032\\
3.35	0.0260469834641583\\
3.36	0.0255260024609253\\
3.37	0.0250154420129733\\
3.38	0.0245150936861919\\
3.39	0.0240247532157841\\
3.4	0.0235442204228596\\
3.41	0.0230732991326991\\
3.42	0.0226117970946497\\
3.43	0.0221595259036324\\
3.44	0.0217163009232065\\
3.45	0.021281941210184\\
3.46	0.0208562694407514\\
3.47	0.0204391118380599\\
3.48	0.0200302981012792\\
3.49	0.0196296613360577\\
3.5	0.0192370379863802\\
3.51	0.0188522677677892\\
3.52	0.0184751936019405\\
3.53	0.0181056615524671\\
3.54	0.0177435207621285\\
3.55	0.017388623391216\\
3.56	0.0170408245571934\\
3.57	0.0166999822755373\\
3.58	0.0163659574017732\\
3.59	0.0160386135746566\\
3.6	0.0157178171605058\\
3.61	0.0154034371986353\\
3.62	0.0150953453478912\\
3.63	0.0147934158342474\\
3.64	0.0144975253994576\\
3.65	0.0142075532507294\\
3.66	0.013923381011409\\
3.67	0.0136448926726471\\
3.68	0.0133719745460393\\
3.69	0.0131045152172071\\
3.7	0.0128424055003105\\
3.71	0.0125855383934713\\
3.72	0.0123338090350864\\
3.73	0.0120871146610153\\
3.74	0.011845354562626\\
3.75	0.0116084300456774\\
3.76	0.0113762443900257\\
3.77	0.0111487028101375\\
3.78	0.0109257124163902\\
3.79	0.0107071821771519\\
3.8	0.0104930228816112\\
3.81	0.0102831471033605\\
3.82	0.0100774691647005\\
3.83	0.00987590510166124\\
3.84	0.00967837262972356\\
3.85	0.00948479111022597\\
3.86	0.0092950815174411\\
3.87	0.00910916640631618\\
3.88	0.00892696988085123\\
3.89	0.00874841756311771\\
3.9	0.00857343656289033\\
3.91	0.00840195544789186\\
3.92	0.00823390421462833\\
3.93	0.00806921425981087\\
3.94	0.00790781835234767\\
3.95	0.00774965060589785\\
3.96	0.00759464645197043\\
3.97	0.00744274261356779\\
3.98	0.00729387707934992\\
3.99	0.00714798907831943\\
4	0.00700501905501185\\
4.01	0.0068649086451815\\
4.02	0.00672760065197522\\
4.03	0.00659303902258122\\
4.04	0.00646116882534562\\
4.05	0.00633193622734725\\
4.06	0.0062052884724204\\
4.07	0.00608117385961671\\
4.08	0.00595954172209951\\
4.09	0.0058403424064585\\
4.1	0.00572352725243926\\
4.11	0.00560904857307837\\
4.12	0.00549685963523515\\
4.13	0.00538691464051295\\
4.14	0.0052791687065629\\
4.15	0.00517357784876066\\
4.16	0.00507009896224974\\
4.17	0.00496868980434544\\
4.18	0.00486930897728867\\
4.19	0.00477191591134536\\
4.2	0.00467647084824569\\
4.21	0.0045829348249522\\
4.22	0.00449126965775344\\
4.23	0.00440143792667632\\
4.24	0.00431340296020999\\
4.25	0.00422712882033478\\
4.26	0.00414258028785037\\
4.27	0.00405972284799987\\
4.28	0.00397852267637707\\
4.29	0.0038989466251206\\
4.3	0.00382096220937972\\
4.31	0.00374453759405418\\
4.32	0.00366964158079713\\
4.33	0.00359624359527964\\
4.34	0.00352431367470852\\
4.35	0.00345382245559476\\
4.36	0.0033847411617664\\
4.37	0.00331704159262091\\
4.38	0.00325069611161294\\
4.39	0.00318567763497203\\
4.4	0.0031219596206461\\
4.41	0.00305951605746637\\
4.42	0.0029983214545283\\
4.43	0.00293835083078573\\
4.44	0.00287957970485265\\
4.45	0.00282198408500922\\
4.46	0.00276554045940766\\
4.47	0.00271022578647335\\
4.48	0.00265601748549963\\
4.49	0.00260289342742859\\
4.5	0.00255083192581803\\
4.51	0.002499811727988\\
4.52	0.00244981200634499\\
4.53	0.00240081234987938\\
4.54	0.00235279275583305\\
4.55	0.00230573362153386\\
4.56	0.00225961573639298\\
4.57	0.0022144202740625\\
4.58	0.00217012878475059\\
4.59	0.00212672318768897\\
4.6	0.00208418576375221\\
4.61	0.0020424991482243\\
4.62	0.00200164632370974\\
4.63	0.00196161061318666\\
4.64	0.00192237567319878\\
4.65	0.00188392548718341\\
4.66	0.00184624435893331\\
4.67	0.00180931690618881\\
4.68	0.0017731280543584\\
4.69	0.00173766303036499\\
4.7	0.00170290735661512\\
4.71	0.00166884684508866\\
4.72	0.00163546759154699\\
4.73	0.001602755969857\\
4.74	0.00157069862642819\\
4.75	0.00153928247476199\\
4.76	0.00150849469010885\\
4.77	0.00147832270423329\\
4.78	0.00144875420028302\\
4.79	0.00141977710776092\\
4.8	0.00139137959759752\\
4.81	0.00136355007732223\\
4.82	0.00133627718633075\\
4.83	0.00130954979124752\\
4.84	0.00128335698138081\\
4.85	0.00125768806426867\\
4.86	0.00123253256131391\\
4.87	0.00120788020350668\\
4.88	0.00118372092723219\\
4.89	0.00116004487016254\\
4.9	0.00113684236723084\\
4.91	0.00111410394668548\\
4.92	0.00109182032622354\\
4.93	0.00106998240920186\\
4.94	0.00104858128092291\\
4.95	0.0010276082049964\\
4.96	0.00100705461977211\\
4.97	0.000986912134845453\\
4.98	0.000967172527631786\\
4.99	0.000947827740010168\\
5	0.000928869875033615\\
};
\addlegendentry{\(u\)};

\end{axis}
\end{tikzpicture}%
    \caption{Stellgröße \(u(\cdot)\) über die Zeit.}
    \label{fig:timeseries_input}
\end{subfigure}

\caption{Beispiel für Exporte  aus MATLAB mit \texttt{matlab2tikz}.}
\label{fig:timeseries}
\end{figure}


\begin{figure}[htbp]
    \centering
    \begin{tikzpicture}
\begin{axis}[
domain=-2:2,
xmin = -2,
xmax = 2,
x label style={at={(axis description cs:1.1,-0.01)},anchor=north},
xlabel = {$x_1$},
ymin = -2,
ymax = 2,
ylabel = {$x_2$},
ylabel style={rotate=-90},
view={0}{90},
axis background/.style={fill=white},
legend style={at={(0.8,0.8)},anchor=south}
]
    \addplot3[UStuttLightBlue,
    quiver={
        u={y/2},
        v={(.4+y/2)*(x<-y-0.1) +(y/2-.4)*(x>-y+.1) -y/2},
        scale arrows=0.3,
    },
    -stealth,samples=12]
    {0-x-y};
    \addlegendentry{Trajektorie abhängig vom Zustand};

    \addplot[UStuttDarkGreen, thick] coordinates{(-2, 2)
                                    (2, -2)};
    \addlegendentry{Asypmtotische Gerade $s(\bm x) = 0 \,\,(x_1 = -\lambda x_2)$};
\end{axis}
\end{tikzpicture}
    \caption{Phasenplot}
    \label{fig:phasenplot}
\end{figure}

% alle Floats werden vor dieser Zeile ausgegeben
\FloatBarrier

\section{Code}

Für Code-Ausschnitte eignet sich z.B. das Paket \texttt{listings}. 
Achten Sie darauf, nur absolut notwendigen Code aufzunehmen, \autoref{lst:bad_code_example} ist ein Negativbeispiel.

\begin{lstlisting}[language=C++, caption={Dieser Code eignet sich nicht zum Aufführen in Ihrer Arbeit.}, label={lst:bad_code_example}]
#include <iostream>

using namespace std;

int main(void){
    cout << "Das ist kein gutes Beispiel." << endl;
    return 0;
}
\end{lstlisting}

\section{Abkürzungen}

Für Abkürzungen lässt sich das Paket \texttt{acronym} verwenden. 

Der Befehl \texttt{\textbackslash ac\{\}} führt bei der ersten Verwendung die Abkürzung ein, beispielsweise das \ac{ISW} und mehrere \acp{SPS}. Danach wird automatisch die Kurzform verwendet, also \ac{ISW} und \acp{SPS}. Die Definition der Abkürzungen kann in einer eigenen Datei erfolgen, das Beispieldokument bindet \texttt{chapters/Acronyms} ein.

Nach der Kurzfassung empfiehlt es sich, die Abkürzungen mit \texttt{\textbackslash acreset} zurückzusetzen, damit sie neu eingeführt werden.



% Dieser Befehl ist nur zum Füllen des Literaturverzeichnis in diesem Beispiel und soll in Ihrer Arbeit nicht verwendet werden.
\nocite{*}